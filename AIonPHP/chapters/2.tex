\section{Sejarah Machine Learning}
Sejak pertama kali komputer diciptakan manusia sudah memikirkan bagaimana caranya agar komputer dapat belajar dari pengalaman. Hal tersebut terbukti pada tahun 1952, Arthur Samuel menciptakan 
sebuah program, game of checkers, pada sebuah komputer IBM. Program tersebut dapat mempelajari gerakan untuk memenangkan permainan checkers dan menyimpan gerakan tersebut kedalam memorinya.
Istilah machine learning pada dasarnya adalah proses komputer untuk belajar dari data (learn from data). Tanpa adanya data, komputer tidak akan bisa belajar apa-apa. Oleh karena itu jika kita ingin belajar machine learning, pasti akan terus berinteraksi dengan data. Semua pengetahuan machine learning pasti akan melibatkan data. Data bisa saja sama, akan tetapi algoritma dan pendekatan nya berbeda-beda untuk mendapatkan hasil yang optimal.
\begin{enumerate}
	\item pembelajaran terarah (Supervised Learning)
	\item pembelajaran tak terarah (Unsupervised Learning)
	\item Pembelajaran semi terarah (Semi-supervised Learning)
	\item Reinforcement Learning
\end{enumerate}
\section{Dampak Machine Learning di Masyarakat}
Penerapan teknologi machine learning mau tidak mau pasti telah dirasakan sekarang. Setidaknya ada dua dampak yang saling bertolak belakang dari pengembangan teknolgi machine learning. Ya, dampak positif dan dampak negatif.Salah satu dampak positif dari machine learning adalah menjadi peluang bagi para wirausahawan dan praktisi teknologi untuk terus-menerus berkarya dalam mengembangkan sebuah bidang teknologi machine learning. Terbantunya aktivitas yang harus dilakukan manusia pun menjadi salah satu dampak positif machine learning. Sebagai contohnya adalah adanya fitur pengecekan ejaan untuk tiap bahasa pada Microsoft Word. Pengecekan secara manual akan memakan waktu berhari-hari dan melibatkan banyak tenaga untuk mendapatkan penulisan yang sempurna. Tapi dengan bantuan fitur pengecekan ejaan tersebut, secara real-time kita bisa melihat kesalahan yang terjadi pada saat pengetikan.
Akan tetapi disamping itu ada dampak negatif yang harus kita waspadai. Adanya pemotongan tenaga kerja karena pekerjaan telah digantikan oleh alat teknologi machine learning adalah suatu permasalahan yang harus dihadapi. Ditambah dengan ketergantungan terhadap teknologi akan semakin terasa. Manusia akan lebih terlena oleh kemampuan gadget-nya sehingga lupa belajar untuk melakukan suatu aktivitas tanpa bantuan teknologi.
\section{Deep Learning}
Dalam istilah praktis, deep learning merupakan bagian dari machine learning. Sebuah model machine learning perlu 'diberitahu' untuk bagaimana ia menciptakan prediksi akurat, dengan terus diberikan data. Sementara model deep learning dapat mempelajari metode komputasinya sendiri, dengan 'otaknya' sendiri, apabila diibaratkan.Sebuah model deep learning dirancang untuk terus menganalisis data dengan struktur logika yang mirip dengan bagaimana manusia mengambil keputusan. Untuk dapat mencapai kemampuan itu, deep learning menggunakan struktur algoritma berlapis yang disebut artificial neural network (ANN). Dikutip dari Zendeks, desain ANN terinspirasi dari jaringan neural biologis dari otak manusia. Hal ini membuat mesin kecerdasannya menjadi jauh lebih tangguh dibandingkan model machine learning standar. Rumit memang untuk memastikan model deep learning yang diciptakan tidak memberikan kesimpulan yang tidak tepat. Tapi ketika ia telah bekerja dengan benar, maka fungsi deep learning akan menjadi terobosan yang berpotensi menjadi tulang belakang sebuah kecerdasan buatan sebenarnya. Data-data yang digunakan dalam sebuah deep learning sangatlah penting, karena semakin banyak datanya, maka semakin banyak yang bisa dipahami model deep learning tersebut. Contoh dari penggunaan model deep learning bisa dilihat dari AlphaGo-nya. Google menciptakan program komputer yang belajar bermain sebuah game sejenis catur dari China bernama Go. Tentunya, game ini membutuhkan pemikiran dan intuisi yang tajam untuk menang. Dengan bermain melawan pemain Go profesional, deep learning AlphaGo mempelajari bagaimana ia bermain di tingkat yang belum terjamah sebelumnya dalam kecerdasan buatan. Hebatnya, apa yang dilakukannya tanpa instruksi apapun ketika melancarkan gerakan-gerakan spesifik. Saat si pemain AlphaGo berhasil mengalahkan sejumlah pemain Go 'nyata' dunia, dunia melihat bagaimana cerdasnya sebuah mesin yang bahkan bisa mengungguli manusia.
\section{Teknologi}
Teknologi adalah berbagai keperluan serta sarana berbentuk aneka macam peralatan atau sistem yang berfungsi untuk memberikan kenyamanan serta kemudahan bagi manusia.Teknologi berasal dari kata technologia (bahasa Yunani) techno artinya ‘keahlian’ danlogia artinya ‘pengetahuan’. Pada awalnya makna teknologi terbatas pada benda- benda berwujud seperti peralatan- peralatan atau mesin. Perkembangan teknologi adalah perubahan sistematis yang terjadi terhadap teknologi. Selama beri-ribu tahun lalu teknologi sudah dikenal oleh manusia, hanya saja bentuk- bentuknya tidak secanggih dengan apa yang kita temukan di masa kini.
\begin{enumerate}
	\item Masa pra-Sejarah
Pada masa pra sejarah ini, teknologi yang digunakan terbuat dari batu, perunggu dan besi. Teknologi yang dikenal di zaman pra-sejarah contohnya adalah Pedang, kapak genggam dan bejana perunggu.
	\item Teknologi Jaman Kuno
Pada masa ini teknologi sudah berkembang ke arah kontruksi, maritim, pertanian dan alat- alat tulis. Manusia sudah mengenal bagaimana membangun sebuah kontruksi bangunan sampai pada tahap rumit. Contohnya Piramid, Kapal, Mercusuar dan jam matahari.
	\item Teknologi Abad Pertengahan Hingga era Modern
Pada masa ini teknologi yang digunakan sudah mulai mengalami kemajuan, hal ini ditandai dengan adanya berbagai penemuan, seperti di bidang astronomi, medis, matematika, militer hingga ilmu kartografi. Contohnya busur silang, mesin cetak, aljabar dan navigasi kapal.
	\item Teknologi era Revolusi Industri
Perkembangan teknologi mulai terlihat semakin jelas di masa ini. Berbagai jenis mesin berhasil dibuat yang kemudian menggantikan tenaga manusia menjadi tenaga mesin. Masa ini adalah cikal bakal perkembangan teknologi di masa kini. Contohnya Mobil generasi awal, telegrap, telepon, mesin tenun, mesin uap dan sepeda.
	\item Teknologi di Abad 20
Pada masa ini. Neil Amstrong berhasil mendarat di bulan. Teknologi dalam bidang lain pun berkembang pesat. Dalam bidang militer, bom atom berhasil diciptakan. Transistor yang menjadi cikal bakal ukuran komputer kecil seperti sekarang ini juga ditemukan. Pada akhir abad ini Internet mulai diperkenalkan untuk umum dan komersil. Contoh teknologi lain Abad 20: Kulkas, Teknologi vaksinasi, vakum, microwave.
	\item Perkembangan Teknologi Abad 21
Pada masa ini, berbagai teknologi sudah mulai dikembangkan. Mulai dari teknologi yang dibutuhkan untuk rumah tangga, pendidikan, sosial, teknologi informasi, dan hal lainnya perkembangan dapat dilihat dari aneka inovasi teknologi yang ada saat ini. Kemajuan teknologi menyentuh berbagai macam sektor, mulai dari :
\end{enumerate}
\section{Teknologi Dalam Bidang Ekonomi}
Kemajuan teknologi di bidang ekonomi ini berupa perkembangan sistem keuangan yang digunakan. Jika dahulu orang melakukan bertransaksi secara real atau nyata, atau berhadapan antara pembeli dengan penjual, maka kini beralih menjadi online. Selain itu, sistem keuangan juga jadi berubah menjadi e-money.
\section{Teknologi Pangan}
Sistem pertanian yang ada saat ini tentunya berbeda dengan sistem pertanian pada zaman dahulu, mulai dari bibir, sistem tanam, serta teknik menanamnya.
\section{Teknologi Informasi}
Kemajuan informasi ini ditandai dengan mudahnya masyarakat dalam memperoleh atau mendapatkan informasi melalui internet dengan berbagai perangkat teknologi yang ada.
\section{Teknologi Komunikasi}
Kemajuan komunikasi ini ditandai dengan mudahnya seseorang untuk berkomunikasi dengan orang lain, walau dengan jarak yang cukup jauh.
\section{Teknologi Transportasi}
Salah satu kemajuan dalam bidang transportasi ini adalah adanya berbagai macam alat transportasi modern, yang mempermudah seseorang untuk mengangkut barang atau bepergian dari 1 temat ke tempat lain dengan mudah
\section{Teknologi Medis}
Salah satu kemajuan dalam dunia medis ini adalah ditemukannya berbagai macam vaksin guna mencegah berbagai macam penyakit berbahaya.
\section{Teknologi Pendidikan}
Adapun teknologi yang turut berkembang dalam dunia pendidikan adalah, berkembangnya sistem pendidikan jadi lebih baik, tenaga pendidik serta murid mudah memahami berbagai pelajaran yang diberikan, dll.






/section{ML App/Technology}
Sejak pertama kali komputer diciptakan manusia sudah memikirkan bagaimana caranya agar komputer dapat belajar dari pengalaman. Hal tersebut terbukti pada tahun 1952, Arthur Samuel menciptakan sebuah program, game of checkers, pada sebuah komputer IBM. Program tersebut dapat mempelajari gerakan untuk memenangkan permainan checkers dan menyimpan gerakan tersebut kedalam memorinya.
Istilah machine learning pada dasarnya adalah proses komputer untuk belajar dari data (learn from data). Tanpa adanya data, komputer tidak akan bisa belajar apa-apa. Oleh karena itu jika kita ingin belajar machine learning, pasti akan terus berinteraksi dengan data. Semua pengetahuan machine learning pasti akan melibatkan data. Data bisa saja sama, akan tetapi algoritma dan pendekatan nya berbeda-beda untuk mendapatkan hasil yang optimal.
/subsection{Apa itu Machine Learning?}
Machine Learning merupakan salah satu cabang dari disiplin ilmu Kecerdasan Buatan (Artificial Intellegence) yang membahas mengenai pembangunan sistem yang berdasarkan pada data. Banyak hal yang dipelajari, akan tetapi pada dasarnya ada 4 hal pokok yang dipelajari dalam machine learning.
Pembelajaran Terarah (Supervised Learning)
Pembelajaran Tak Terarah (Unsupervised Learning)
Pembelajaran Semi Terarah (Semi-supervised Learning)
Reinforcement Learning
/subsection{Mengetahui Machine Learning}
Untuk mengetahui lebih lengkap tentang Machine Learning, kawan-kawan bisa mengikuti course dengan instruktur profesor Andrew NG dari Stanford University.
Aplikasi Machine Learning
Contoh penerapan machine learning dalam kehidupan adalah sebagai berikut.
Penerapan di bidang kedoteran contohnya adalah mendeteksi penyakit seseorang dari gejala yang ada. Contoh lainnya adalah mendeteksi penyakit jantung dari rekaman elektrokardiogram.
Pada bidang computer vision contohnya adalah penerapan pengenalan wajah dan pelabelan wajah seperti pada facebook. Contoh lainnya adalah penterjemahan tulisan tangan menjadi teks.
Pada biang information retrival contohnya adalah penterjemahan bahasa dengan menggunakan komputer, mengubah suara menjadi teks, dan filter email spam.
/subsection{Dampak Machine Learning bagi Masyarakat}
Penerapan teknologi machine learning mau tidak mau pasti telah dirasakan sekarang. Setidaknya ada dua dampak yang saling bertolak belakang dari pengembangan teknolgi machine learning. Ya, dampak positif dan dampak negatif.
Salah satu dampak positif dari machine learning adalah menjadi peluang bagi para wirausahawan dan praktisi teknologi untuk terus berkarya dalam mengembangkan teknologi machine learning. Terbantunya aktivitas yang harus dilakukan manusia pun menjadi salah satu dampak positif machine learning. Sebagai contohnya adalah adanya fitur pengecekan ejaan untuk tiap bahasa pada Microsoft Word. Pengecekan secara manual akan memakan waktu berhari-hari dan melibatkan banyak tenaga untuk mendapatkan penulisan yang sempurna. Tapi dengan bantuan fitur pengecekan ejaan tersebut, secara real-time kita bisa melihat kesalahan yang terjadi pada saat pengetikan.
/begin{enumerate}
/item Model: melakukan suatu aktivitas tanpa bantuan teknologi.
/item Parameter: tanpa adanya dukungan dari teknologi
/item Pemelajaran: sistem yang menyesuaikan peran fungsi pada machine learning
/end{enumerate}
Subsection{ Pengertian Teknologi}
Apa yang dimaksud dengan teknologi (technology)? Secara garis umum pengertian dari sebuah  teknologi adalah ilmu pengetahuan yang mempelajari tentang keterampilan dalam menciptakan alat, metode pengolahan, dan ekstraksi benda, untuk membantu menyelesaikan berbagai permasalahan dan pekerjaan manusia sehari-hari.
Ada juga yang menyebutkan bahwa arti teknologi adalah semua sarana dan prasarana yang diciptakan oleh manusia untuk menyediakan berbagai barang yang dibutuhkan bagi keberlangsungan dan kenyamanan hidup manusia itu sendiri. Secara etimologis, kata “teknologi” berasal dari bahasa Yunani, yaitu “technologia” dimana kata tech berarti keahlian dan logia berarti pengetahuan.
Dulunya makna teknologi hanya terbatas pada benda-benda yang memiliki wujud, misalnya mesin dan peralatan. Namun makna teknologi mengalami perluasan dan tidak hanya terbatas pada benda berwujud saja tapi juga benda yang tidak berwujud, misalnya metode, ilmu pengetahuan, software, dan lain-lain. Sehingga pengertian teknologi adalah suatu cara, proses, alat, mesin, kegiatan ataupun gagasan yang dibuat untuk mempermudah berbagai kegiatan manusia.
/begin{itemize}
/item Iskandar Alisyahbana
Menurut Iskandar Alisyahbana, pengertian teknologi adalah cara melakukan sesuatu untuk memenuhi kebutuhan manusia dengan bantuan alat dan akal, sehingga seakan-akan memperpanjang, memperkuat, atau mebuat lebih ampuh anggota tubuh, pancaindra, dan otak manusia.
/item Manuel Castells
Menurut Manuel Castells, pengertian teknologi adalah suatu alat, aturan, dan prosedur penerapan pengetahuan ilmiah untuk pekerjaan tertentu dalam kondisi yang dapat memungkinkan pengulangan.
/item Gary J. Anglin
Menurut Gary J. Anglin, arti teknologi adalah penerapan ilmu-ilmu perilaku dan alam serta pengetahuan lain secara bersistem dan mensistem untuk memecahkan masalah manusia.
/item Jacques Ellil
Menurut Jacques Ellil, pengertian teknologi adalah keseluruhan metode yang secara rasional mengarah dan memiliki ciri efisiensi dalam setiap kegiatan manusia.
/item Merriam Webster
Menurut Merriam Webster, pengertian teknologi adalah penerapan pengetahuan praktis khususnya di bidang tertentu; cara menyelesaikan tugas terutama menggunakan proses teknis, metode, atau pengetahuan; dan aspek khusus dari bidang usaha tertentu.
/end{enumerate}
Subsection {Jenis-Jenis Teknologi}
/item Teknologi Informasi
Teknologi informasi (TI) adalah suatu teknologi yang dapat membantu manusia dalam menyampaikan informasi kepada orang lain dalam waktu yang cepat. Beberapa produk yang termasuk dalam teknologi ini diantaranya; televisi, radio, media online, dan lainnya.
/item Teknologi Komunikasi
Teknologi komunikasi adalah suatu teknologi yang dapat membantu manusia dalam berkomunikasi satu sama lain dan saling mengirimkan informasi dengan menggunakan suatu perangkat khusus. Beberapa produk yang termasuk dalam teknologi komunikasi diantaranya; smartphone, mesin fax, email, aplikasi chatting, dan lainnya.
/item Teknologi Transportasi
Teknologi transportasi adalah suatu teknologi yang membantu manusia untuk berpindah tempat (transportasi) dari suatu lokasi ke lokasi lainnya dalam waktu yang cepat. Beberapa produk yang termasuk dalam teknologi ini diantaranya; kereta listrik, mobil, pesawat, kapal laut.
/item Teknologi Pendidikan
Teknologi pendidikan adalah teknologi yang berhubungan dengan dunia pendidikan dimana kegiatannya memanfaatkan alat bantu tertentu. Beberapa yang termasuk dalam teknologi pendidikan diantaranya; metode pengajaran terbaru, peralatan laboratorium sekolah, komputer, OHP, dan lainnya.
/item Teknologi Medis
Teknologi medis adalah suatu teknologi yang berkaitan dengan dunia kedokteran dimana kegiatan medis sudah memanfaatkan teknologi komputer. Beberapa produk yang termasuk dalam teknologi medis diantaranya; tensimeter, termoter tubuh, stetoskop, alat suntik dan infus, alat USG, alat X-Ray, dan lainnya.
/item Teknologi Konstruksi
Teknologi konstruksi adalah suatu teknologi yang berhubungan dengan struktur bangunan. Beberapa yang termasuk di dalamnya diantaranya; metode kerja, software gambar struktur (AutoCAD), alat-alat berat, dan lain sebagainya.


